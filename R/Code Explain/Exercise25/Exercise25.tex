\documentclass{article}
\title{Project Euler - Exercise 25}
% \author{Caio Assumpção Rezzadori}
\date{}
\usepackage{amsmath}
\usepackage{amssymb}
\usepackage{mathtools}

\DeclarePairedDelimiter\floor{\lfloor}{\rfloor}
\DeclarePairedDelimiter\ceil{\lceil}{\rceil}

\begin{document}
\maketitle

\section*{The problem}
The Fibonacci Sequence is given by $f_n = f_{n - 1} + f_{n - 2}$
where\\ $f_0= f_1 = 1$.\\

What is the index of the first term in the Fibonacci sequence to contain 
$d$ digits?

\section*{Iterative solution}

The algorithm with iterations it's straight forward:

\subsection*{The algorithm}
\begin{enumerate}
    \item $n \leftarrow 3$\\
    $f_{n - 2} \leftarrow 1$\\
    $f_{n - 1} \leftarrow 1$\\
    $f_n \leftarrow f_{n - 2} + f_{n - 1}$
    \item While $f_n < 10^{d - 1}$, \\
    $ f_{n-2} \leftarrow f_{n - 1}$,\\
    $f_{n-2} \leftarrow f_{n - 1}$, \\
    $f_n \leftarrow f_{n - 2} + f_{n - 1}$,
    $n \leftarrow n + 1$
\end{enumerate}

\section*{Solving with Linear Algebra}

If we define the linear transformation above
\begin{equation*}
    T =
    \begin{bmatrix}
        0 & 1\\
        1 & 1
    \end{bmatrix}
\end{equation*}

We notice that, if we apply it to the vector $\begin{bmatrix}
    f_{n - 2} \\
        f_{n - 1}
\end{bmatrix}$, we have
\begin{equation*}
    \Rightarrow
    T \begin{bmatrix}
        f_{n - 2} \\
        f_{n - 1}
    \end{bmatrix} =
    \begin{bmatrix}
        f_{n - 1} \\
        f_{n - 1} + f_{n - 2}
    \end{bmatrix} =
    \begin{bmatrix}
        f_{n - 1} \\
        f_{n}
    \end{bmatrix}
\end{equation*}

Starting with the vector $v_0 = \begin{bmatrix}
    f_{0} \\
        f_{1}
\end{bmatrix}$ = $\begin{bmatrix}
    1 \\
    1
\end{bmatrix}$, and successively applying the transformation to the new result,
we have:

\begin{equation*}
    \begin{aligned}
    Tv_0 = T\begin{bmatrix}
        1 \\
        1
    \end{bmatrix} = \begin{bmatrix}
        1\\
        2
    \end{bmatrix}=\begin{bmatrix}
        f_1\\
        f_2
    \end{bmatrix} = v_1\\
    \Rightarrow
    Tv_1 = T(Tv_0) = T^2 v_0 =\begin{bmatrix}
        2\\
        3
    \end{bmatrix} = \begin{bmatrix}
        f_2\\
        f_3
    \end{bmatrix}  = v_2 \\
    \Rightarrow
    Tv_2 = T(T^2 v_0) = T^3 v_0 =\begin{bmatrix}
        3\\
        5
    \end{bmatrix} = \begin{bmatrix}
        f_3\\
        f_4
    \end{bmatrix}  = v_3
    \end{aligned}
\end{equation*}

If we continue the iterations, it's easy to notice that the pattern holds.\\
Then, we have $T^n v_0 = v_n = \begin{bmatrix}
    f_n\\
    f_{n + 1}
\end{bmatrix}$. This relation means that one way to calculate the $n$'th term
of the Fibonacci Sequence is computing $T^{n - 1}$ and apply to $v_0$.

Since multiplication of matrices is a operation with high computational cost,
we want to find a way to compute $T^n$ with more efficiency.

Let's calculate the eigenvalues of $T$, i.e, the scalars
$\lambda \in \mathbb{C}$ and $v \in \mathbb{R}^2$ such that:
    \begin{equation}
        Tv = \lambda v
    \end{equation}
\begin{equation*}
    \begin{aligned}
        \Rightarrow Tx - \lambda v = 0\\
        \Rightarrow (T - \lambda I)v = 0
    \end{aligned}
\end{equation*}


Since we want non trivial solutions, we want that the matrix $T - \lambda I$ be
singular (not invertible). Then:
\begin{equation*}
    \begin{aligned}
    \det(T - \lambda I) = 0\\
    \det(\begin{bmatrix}
        0 & 1 \\
        1 & 1
    \end{bmatrix} - \begin{bmatrix}
        \lambda & 0\\
        0 & \lambda
    \end{bmatrix})\\
    =\det(\begin{bmatrix}
        -\lambda & 1\\
        1 & 1 - \lambda
    \end{bmatrix})\\
    = \lambda^2 - \lambda - 1 = 0
    \end{aligned}
\end{equation*}

Then, the eigenvalues of $T$ are $\lambda_1 = \dfrac{1 + \sqrt{5}}{2}$ and
$\lambda_2 = \dfrac{1 - \sqrt{5}}{2}$. Since the eigenvalues are distinct, the
eigenvectors $v_1, v_2$ associated are linearly independent.

Let $V = [v_1 , v_2]$ the matrix where the eigenvectors are the columns and $\Lambda = \begin{bmatrix}
    \lambda_1 & \\
    & \lambda_2
\end{bmatrix}$ the diagonal matrix of eigenvalues.

We can write the equation $(1)$ as:
\begin{equation*}
    \begin{aligned}
        TV = V\Lambda\\
        \Rightarrow T = V \Lambda V^{-1}
    \end{aligned}
\end{equation*}

since $V$ is invertible (linearly independent eigenvectors).

Notice now, that if we want to calculate powers of $T$, we have:

\begin{equation*}
    \begin{aligned}
        T^2 = (V\Lambda V^{-1})(V\Lambda V^{-1})\\
        = V\Lambda (VV^{-1})\Lambda V^{-1}\\
        = V \Lambda I \Lambda V^{-1} \\
        =V \Lambda^2 V^{-1}
    \end{aligned}
\end{equation*}

This means, that if we want to compute $T^n$, we just have to compute \\
$\Lambda ^n = \begin{bmatrix}
    \lambda_1 ^n & \\
    & \lambda_2 ^n
\end{bmatrix}$.

This means that the $n$'th term of the Fibonacci Sequence is given by
\begin{equation*}
    \begin{bmatrix}
        f_{n - 1}\\
        f_n
    \end{bmatrix} = V \begin{bmatrix}
        \lambda_1 ^{n - 1} & \\
    & \lambda_2 ^{n - 1}
    \end{bmatrix}V^{-1}
\end{equation*}

A important observation, is since $T$ is a symmetric matrix, we know that $V^{-1} = V^T$.

\newpage
For this particular transformation, it can be shown that the following 
inequality holds:

\begin{equation}
    f_n \leq 1.5 \lambda_1 ^{n - 1}
\end{equation}

To demonstrate that, involves some messy calculations using the coordinates of
its eigenvectors, so we are not going to explain that.

Then, one heuristic to find the first element of the Fibonacci Sequence with
$d \in \mathbb{N}$ digits is using $(2)$.

Let's calculate the $n$'th iteration which most approximates $f_n$ to $10^{d - 1}$:

\begin{equation*}
    \begin{aligned}
        f_n \leq 1.5\lambda_1 ^{n - 1} \approx 10^{d-1}\\
        \Rightarrow \log_{10}(1.5\lambda_1 ^{n - 1}) \approx
                                            \log_{10}(10^{d-1})\\
        \Rightarrow \log_{10}(1.5) + (n - 1) \log_{10}(\lambda_1)
                                    \approx d - 1\\
        \Rightarrow n - 1 \approx \dfrac{d - 1 - \log_{10}(1.5)}{\log_{10}(\lambda_1)}\\
        \Rightarrow n  \approx {\dfrac{d - 1 - \log_{10}(1.5)}{\log_{10}(\lambda_1)} + 1}
    \end{aligned}
\end{equation*}

Then, if we want to calculate the first element of the Fibonacci Sequence which
has $d$ digits, a good initial value to start the iterations is
given by
\begin{equation*}
    n_0 = \floor*{\dfrac{d  - 1- \log_{10}(1.5)}{\log_{10}(\lambda_1)} + 1}
\end{equation*}

where $\floor{.}$ is the floor function.\\

This choose of $n$ guarantees that
$f_n \leq 1.5\lambda_1 ^{n - 1} \leq 10^{d - 1}$.\\

\newpage
\section*{The algorithm}

Therefore, the algorithm to solve the problem, is given by:
\begin{enumerate}
    \item Calculate the Spectral Decomposition of $T = \begin{bmatrix}
        0 & 1\\
        1 & 1
    \end{bmatrix}$, i.e, calculating the eigenvalues and eigenvector matrices,
    represented by $\Lambda$ and $V$ respectively;
    \item Compute $n_0 = \floor*{\dfrac{d  - 1- \log_{10}(1.5)}{\log_{10}(\lambda_1)} + 1}$,
    where  $\lambda_1$ is the greatest eigenvalue of $\Lambda$;
    \item Let $n \leftarrow n_0$ and compute:
    \begin{equation*}
        \begin{bmatrix}
            f_{n-1}\\
            f_n
        \end{bmatrix} \leftarrow T^{n-1}v_0 = V \Lambda^{n-1} V^{-1} v_0
    \end{equation*} where $v_0 = \begin{bmatrix}
        1 \\
        1
    \end{bmatrix}$;
    \item While $f_n < 10^{d - 1}$, $\begin{bmatrix}
        f_{n-1}\\
        f_{n}
    \end{bmatrix} \leftarrow \begin{bmatrix}
        f_n\\
        f_{n+1}
    \end{bmatrix}$ and $n \leftarrow n + 1$.
\end{enumerate}
\vspace{1cm}

This should give the desired result.\\

Since the matrix $T$ has dimension $2 \times 2$, which is fixed, then
the operation of multiplication of matrices may have constant time, just as
$\log_{10}$ and power of numbers. And since the only step with a loop is
$(4.)$, which is really short because we estimated the initial value really
close to the result, the algorithm should have a complexity really close to $O(1)$.
% By the \textbf{Spectral Theorem}, since $T$ is a symmetric matrix, we can
% write it as $T = V \Lambda V^{-1}$, where $\Lambda$ is the diagonal matrix of its
% eigenvalues and $V$ the matrix of its eigenvectors.




\end{document}