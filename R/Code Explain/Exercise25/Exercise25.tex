\documentclass{article}
\title{Project Euler - Exercise 25}
% \author{Caio Assumpção Rezzadori}
\date{}
\usepackage{amsmath}
\usepackage{amssymb}
\usepackage{mathtools}

\DeclarePairedDelimiter\floor{\lfloor}{\rfloor}

\begin{document}
\maketitle
The Fibonacci Sequence is given by $f_n = f_{n - 1} + f_{n - 2}$
where\\ $f_0= f_1 = 1$.
If we define the linear transformation above
\begin{equation*}
    T =
    \begin{bmatrix}
        0 & 1\\
        1 & 1
    \end{bmatrix}
\end{equation*}

We notice that, if we apply it to the vector $\begin{bmatrix}
    f_{n - 2} \\
        f_{n - 1}
\end{bmatrix}$, we have
\begin{equation*}
    \Rightarrow
    T \begin{bmatrix}
        f_{n - 2} \\
        f_{n - 1}
    \end{bmatrix} =
    \begin{bmatrix}
        f_{n - 1} \\
        f_{n - 1} + f_{n - 2}
    \end{bmatrix} =
    \begin{bmatrix}
        f_{n - 1} \\
        f_{n}
    \end{bmatrix}
\end{equation*}

Starting with the vector $v_0 = \begin{bmatrix}
    f_{0} \\
        f_{1}
\end{bmatrix}$ = $\begin{bmatrix}
    1 \\
    1
\end{bmatrix}$, and successively applying the transformation to the new result,
we have:

\begin{equation*}
    \begin{aligned}
    Tv_0 = T\begin{bmatrix}
        1 \\
        1
    \end{bmatrix} = \begin{bmatrix}
        1\\
        2
    \end{bmatrix}=\begin{bmatrix}
        f_1\\
        f_2
    \end{bmatrix} = v_1\\
    \Rightarrow
    Tv_1 = T(Tv_0) = T^2 v_0 =\begin{bmatrix}
        2\\
        3
    \end{bmatrix} = \begin{bmatrix}
        f_2\\
        f_3
    \end{bmatrix}  = v_2 \\
    \Rightarrow
    Tv_2 = T(T^2 v_0) = T^3 v_0 =\begin{bmatrix}
        3\\
        5
    \end{bmatrix} = \begin{bmatrix}
        f_3\\
        f_4
    \end{bmatrix}  = v_3
    \end{aligned}
\end{equation*}

If we continue the iterations, it's easy to notice that the pattern holds.\\
Then, we have $T^n v_0 = v_n = \begin{bmatrix}
    f_n\\
    f_{n + 1}
\end{bmatrix}$. This relation means that one way to calculate the $n$'th term
of the Fibonacci Sequence is computing $T^{n - 1}$ and apply to $v_0$.

Since multiplication of matrices is a operation with high computational cost,
we want to find a way to compute $T^n$ with more efficiency.

Let's calculate the eigenvalues of $T$, i.e, the scalars
$\lambda \in \mathbb{C}$ and $v \in \mathbb{R}^2$ such that:
    \begin{equation}
        Tv = \lambda v
    \end{equation}
\begin{equation*}
    \begin{aligned}
        \Rightarrow Tx - \lambda v = 0\\
        \Rightarrow (T - \lambda I)v = 0
    \end{aligned}
\end{equation*}


Since we want non trivial solutions, we want that the matrix $T - \lambda I$ be
singular (not invertible). Then:
\begin{equation*}
    \begin{aligned}
    \det(T - \lambda I) = 0\\
    \det(\begin{bmatrix}
        0 & 1 \\
        1 & 1
    \end{bmatrix} - \begin{bmatrix}
        \lambda & 0\\
        0 & \lambda
    \end{bmatrix})\\
    =\det(\begin{bmatrix}
        -\lambda & 1\\
        1 & 1 - \lambda
    \end{bmatrix})\\
    = \lambda^2 - \lambda - 1 = 0
    \end{aligned}
\end{equation*}

Then, the eigenvalues of $T$ are $\lambda_1 = \dfrac{1 + \sqrt{5}}{2}$ and
$\lambda_2 = \dfrac{1 - \sqrt{5}}{2}$. Since the eigenvalues are distinct, the
eigenvectors $v_1, v_2$ associated are linearly independent.

Let $V = [v_1 , v_2]$ the matrix where the eigenvectors are the columns and $\Lambda = \begin{bmatrix}
    \lambda_1 & \\
    & \lambda_2
\end{bmatrix}$ the diagonal matrix of eigenvalues.

We can write the equation $(1)$ as:
\begin{equation*}
    \begin{aligned}
        TV = V\Lambda\\
        \Rightarrow T = V \Lambda V^{-1}
    \end{aligned}
\end{equation*}

since $V$ is invertible (linearly independent eigenvectors).

Notice now, that if we want to calculate powers of $T$, we have:

\begin{equation*}
    \begin{aligned}
        T^2 = (V\Lambda V^{-1})(V\Lambda V^{-1})\\
        = V\Lambda (VV^{-1})\Lambda V^{-1}\\
        = V \Lambda I \Lambda V^{-1} \\
        =V \Lambda^2 V^{-1}
    \end{aligned}
\end{equation*}

This means, that if we want to compute $T^n$, we just have to compute \\
$\Lambda ^n = \begin{bmatrix}
    \lambda_1 ^n & \\
    & \lambda_2 ^n
\end{bmatrix}$.

This means that the $n$'th term of the Fibonacci Sequence is given by
\begin{equation*}
    \begin{bmatrix}
        f_{n - 1}\\
        f_n
    \end{bmatrix} = V \begin{bmatrix}
        \lambda_1 ^{n - 1} & \\
    & \lambda_2 ^{n - 1}
    \end{bmatrix}V^{-1}
\end{equation*}

One heuristic to find the first element of the Fibonacci sequence with 1000
digits is using its greatest eigenvalue.

Since the eigenvectors can be normalized,
i.e, $||v_1|| = ||v_2|| = 1$, the value of $f_n$ is going to be most influenced
by $lambda_1^{n - 1}$.
\newpage
One way to estimate this first value, is calculating the which
$n \in \mathbb{N}$ most approximate the relation
$\lambda_1 ^{n - 1} = 10^{1000}$. Let's calculate $n$:
\begin{equation*}
    \begin{aligned}
        \lambda_1 ^{n - 1} \approx 10^{1000}\\
        \Rightarrow \log_{10}(\lambda_1 ^{n - 1}) \approx
                                            \log_{10}(10^{1000})\\
        \Rightarrow (n - 1) \log_{10}(\lambda_1)
                                    \approx 1000\\
        \Rightarrow n = \floor*{\dfrac{1000}{\log_{10}(\lambda_1)} + 1}
    \end{aligned}
\end{equation*}

% By the \textbf{Spectral Theorem}, since $T$ is a symmetric matrix, we can
% write it as $T = V \Lambda V^{-1}$, where $\Lambda$ is the diagonal matrix of its
% eigenvalues and $V$ the matrix of its eigenvectors.




\end{document}